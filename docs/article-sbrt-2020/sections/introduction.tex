\section{Introdução}
    Vacinas tipicamente são armazenadas em refrigeradores encontrados em clínicas, hospitais ou postos de saúde, e necessitam de grande cuidado com relação à sua conservação. Temperaturas abaixo de 2$^\circ$C ou acima de 8$^\circ$C, bem como umidade abaixo de 5\%, alteram a sua qualidade~\cite{ref1}. Dessa maneira, o monitoramento em tempo real desses dados é essencial para evitar a perda de efetividade das vacinas e tem na aplicação soluções baseadas em Internet das Coisa (IoT) uma alternativa para prover sensoriamento remoto dos locais de conservação de vacinas. 
    
    Atualmente, os registros associados as mais diversas tarefas e informações acerca de pacientes, em muitos hospitais e postos de saúde pelo Brasil, são realizadas manualmente, o que pode acarretar em imprecisões na catalogação dessas informações. Esse risco também inclui a informação acerca da qualidade das vacinas disponibilizadas nesses ambientes. Seja no transporte, seja no armazenamento, a falta de garantia na informação disponibilizada põe em risco a garantia de qualidade desses produtos, implicando em risco para a saúde dos usuários.
    
    Nesse contexto, o sistema descrito neste artigo tem como objetivo monitorar continuamente os locais onde as vacinas são armazenadas, por meio dos nós finais que capturam as informações a partir de sensores e enviam para um \textit{gateway}, utilizando tecnologia LoRa. Dessa forma, se em algum momento um conjunto de vacinas não está em boas condições de armazenamento, ou seja, está submetida a temperatura e umidade fora dos padrões estabelecidos pela OMS, os responsáveis por ministrar a vacina aos usuários finais poderão avaliar, de forma mais realista, a qualidade do produto.%, através do relatório que é gerado a um determinado intervalo de tempo.
    
    No entanto, a implementação desta aplicação requer cuidados especiais, visando confiabilidade e segurança no momento de envio e recepção dos pacotes de dados.  Diversas variáveis podem impedir que o envio de informações seja realizado corretamente, como, por exemplo, ausência de alimentação dos nós sensores, desabilitando a aquisição e armazenamento dos dados. %baterias ou pilhas que é utilizada no projeto, posto que, o envio de pacotes pode ser afetado diretamente. 
    
    %de modo que, a tecnologia LoRa permite que o sistema consuma pouca energia.%contenha um esquema de envio de pacotes com baixo custo de energia sobre a bateria.
