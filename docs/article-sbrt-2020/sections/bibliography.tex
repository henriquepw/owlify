\begin{thebibliography}{99}

    \bibitem{ref7} Autoral. Repositorio do Github com os testes realizados.  Disponível em: https://github.com/GComPI-IFPB/lora-transmitter-test. Acessado em 22 de maio, 2020.
    
    \bibitem{ref1} Ministerio da Saude. Manual de Rede de Frio.  Disponível em: http://bvsms.saude.gov.br/bvs/publicacoes/manual\_rede\_frio4ed.pdf. Acessado em 13 de abril, 2020.
    
    \bibitem{ref2} LoRa Alliance. Regional Parameters. Disponível em: https://lora-alliance.org/sites/default/files/2020-01/rp\_2-1.0.0\_final\_release.pdf. Acessado em 15 de abril, 2020.
    
    \bibitem{ref3} Semtech Corporation. Datasheet SX1276/77/78/79. Disponível em: https://www.curtocircuito.com.br/datasheet/modulo/SX1276.pdf. Acessado em 24 de abril, 2020.
    
    \bibitem{ref4} Sandeep Mistry. Documentação da biblioteca LoRa utilizada para Arduíno. Disponível em: https://github.com/sandeepmistry/arduino-LoRa. Acessado em 11 de maio, 2020.
    
    \bibitem{ref5} Universidade Estadual Paulista. Técnicas de Múltiplo Acesso para Redes LORAWAN. Disponível em: https://repositorio.unesp.br/bitstream/handle/11449/156791/000902295.pdf. Acessado em 11 de maio, 2020.
    
   \bibitem{ref6} Thomas Liu. Datasheet DHT22. Disponível em:  https://www.alldatasheet.com/datasheet-pdf/pdf/1132459/ETC2/DHT22.html. Acessado em 18 de maio de 2020.
    
\end{thebibliography}