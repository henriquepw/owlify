\chapter*[Introdução]{Introdução}
\addcontentsline{toc}{chapter}{Introdução}
% ---

A saúde é um fator de suma importância para todos os seres vivos, ele é um problema científico, tecnológico, político, prático e filosófico que refere-se a um estado completo de bem estar físico, emocional, social, intelectual e espiritual\cite{almeida2011saude}. 

Segundo o artigo 196 da Constituição Federal Brasileira a saúde é um direito de todos e dever do Estado garantir medidas políticas sociais e econômicas que visam à diminuição do risco de doenças e de outros agravamentos e ao acesso universal e imparcial às ações e serviços para a sua promoção, proteção e recuperação\cite{de2013direito}.

Para garantirmos nossa saúde, precisamos cuidar do nosso corpo e mente, para isto, uma ferramenta que podemos contar são os imunobiológicos, como as vacinas e os soros, diferente de remédios que ajudam no tratamento de pessoas doentes, as imunobiológicos são uma preparação biológica que fornece imunidade total ou parcial de uma determinada doença autoimune para um indivíduo saudável. As vacinas e os soros se diferem pela sua forma de imunização, as vacinas fornece uma imunização ativa, estimulando o nosso organismo na produção de anticorpos, os soros fornecem uma imunização passiva, provendo os anticorpos para o nosso organismo que foram produzidos  em outros organismo\cite{soma2018tratamento}.

Contudo, os imunobiológicos requerem um cuidado elevado para manter a qualidade e sua eficiência, um dos fatores é que são produtos termolábeis, ou seja, se deterioram após determinado tempo expostos a variações de temperaturas e umidade, portanto, é imprescindível assegurar que seu ambiente de armazenagem mantenha uma temperatura e umidade constante\cite{ministerio2001manual} para garantir uma longevidade maior para o produto. Para este propósito, existem as redes de frio, um processo desenvolvido pelo Programa Nacional de Imunizações, PNI, de conversação, armazenamento e transporte dos medicamentos, objetivando as condições adequadas dos mesmos, mantendo suas características iniciais\cite{ministerio2001manual}.

